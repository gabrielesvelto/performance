\begin{frame}
 \frametitle{Profiling}
 \begin{itemize}
  \item Why profiling?
   \begin{itemize}
    \item To identify hot-spots, regressions and unpredictable issues
    \item To get an accurate idea of the overall performance profile of an
          application and not just a part of it
    \item To make informed decisions on what to optimize
   \end{itemize}
  \item \emph{Never optimize without profiling first!}
 \end{itemize}
\end{frame}

\begin{frame}
 \frametitle{Profiling with the built-in profiler}
 \begin{itemize}
  \item SPS built-in profiler
   \begin{itemize}
    \item The best way to profile anything within Firefox and FxOS
    \item \url{https://developer.mozilla.org/en-US/docs/Performance/Profiling_with_the_Built-in_Profiler}
   \end{itemize}
  \item Advantages
  \begin{itemize}
   \item Captures both native and JavaScript code
   \item Has complementary tools for spotting events (GC, layout, etc)
   \item Profile information is easy to read and share
  \end{itemize}
  \item Disadvantages
  \begin{itemize}
   \item Limited to the main thread
   \item Limited native code analysis in FxOS
   \item Granularity can be too coarse and samples can be skewed
   \item Does not capture system effects
   \item Requires a special build
  \end{itemize}
 \end{itemize}
\end{frame}

\begin{frame}[fragile]
 \frametitle{Profiling on your device}
 \begin{itemize}
  \item Configuring your build
   \begin{itemize}
    \item Start with a regular B2G build
    \item Make sure \sourcecode{elfhack} is disabled in \sourcecode{.mozconfig}
    \item \sourcecode{ac\_add\_options --disable-elf-hack}
   \end{itemize}
  \item Rebuild and flash your device
  \item Start the profiler with \sourcecode{./profile.sh start}
  \item Check for the running applications with \sourcecode{./profile.sh ps}
   \begin{Verbatim}
  PID Name
----- ----------------
 4989 b2g              profiler running
 5037 Usage            profiler running
 5038 Homescreen       profiler running
 5203 (Preallocated a  profiler running
   \end{Verbatim}
 \end{itemize}
\end{frame}

\begin{frame}[fragile]
 \frametitle{Capturing one or more profiles}
 \begin{itemize}
  \item You can capture the profile of an application by specifiying it's name or PID
   \begin{itemize}
    \item \sourcecode{./profile capture Homescreen}
    \item \sourcecode{./profile capture 5038}
   \end{itemize}
  \item If all goes well you'll end up with a .sym file:
   \begin{Verbatim}
Signalling PID: 5038 Homescreen ...
Stabilizing 5038 Homescreen ...
Pulling /data/local/tmp/profile_2_5038.txt into profile_5038_Homescreen.txt
Adding symbols to profile_5038_Homescreen.txt and creating profile_5038_Homescreen.sym ...
Removing old profile files (from device) ... done
   \end{Verbatim}
  \item You can also capture profiles for all processes at the same time by not specifying any parameter
  \begin{itemize}
   \item \sourcecode{./profile capture}
  \end{itemize}
 \end{itemize}
\end{frame}

\begin{frame}
 \frametitle{Analysing your profile}
 \begin{itemize}
  \item The profile will contain up to 100s of the process' activity
  \item Default sample time is 10ms, it can be lowered to 1ms at most
  \item When you have many processes open capturing a profile might kill some
        processes due to OOM so tread carefully
  \item To analyze your profile you will need to upload it to our web-based service Cleopatra
  \begin{itemize}
   \item \url{http://people.mozilla.com/~bgirard/cleopatra/}
   \item It can also be run locally but then you'll lose the ability to share profiles
   \item The code is available here \url{https://github.com/mozilla/cleopatra}
  \end{itemize}
 \end{itemize}
\end{frame}

\begin{frame}
 \begin{center}
  \Huge{Cleopatra demo}
 \end{center}
\end{frame}

\begin{frame}
 \frametitle{Studying a profile}
 \begin{itemize}
  \item Reading call-stacks
  \begin{itemize}
   \item JavaScript code can be easily analyzed and referenced
   \item Native code currently uses markers, it's important to spot major ones
   \begin{itemize}
    \item \sourcecode{nsAppShell::ProcessNextNativeEvent::Wait}
    \item \sourcecode{JS::EvaluateString} and \sourcecode{js::RunScript}
    \item \sourcecode{Timer::Fire}
    \item \sourcecode{nsRefreshDriver::Notify}
    \item \sourcecode{Paint::PresShell::Paint}
    \item \sourcecode{Layout::Flush}
    \item \sourcecode{GC::GarbageCollectNow}
   \end{itemize}
  \end{itemize}
  \item Cleopatra provides hints
   \begin{itemize}
    \item GC markers
    \item Layout markers
    \item I/O markers
   \end{itemize}
 \end{itemize}
\end{frame}

\begin{frame}
 \frametitle{Studying a profile \(continued\)}
 \begin{itemize}
  \item Filtering
  \begin{itemize}
   \item JavaScript-only filtering
   \item Narrow down to a single function/method
  \end{itemize}
  \item Inverting call stacks
  \item Zoom as needed to get a better idea of what's going on
  \item Upload your profile and add a link to your ticket for easy sharing
 \end{itemize}
\end{frame}

\begin{frame}
 \frametitle{Limits and pitfalls}
 \begin{itemize}
  \item Only the main thread is currently sampled, if you have worker threads
        significant time might be spent on them
  \item A lot of activities are delegated to the main B2G process, capturing
        both your app and the b2g process is often a good idea
  \item The profile shows real-time, not CPU time, if the phone is loaded it
        will appear as the app is slower
  \item The profile does not show system activity
  \begin{itemize}
   \item Use \sourcecode{top} to spot high system activity
   \item Watch out for I/O activity, check for \sourcecode{write} or
         \sourcecode{read} calls in your profile, use \sourcecode{top} to
         estimate the wait time
   \item Use \sourcecode{perf} if all else fails
  \end{itemize}
  \item IPC will not show up in the profile so be careful
 \end{itemize}
\end{frame}
